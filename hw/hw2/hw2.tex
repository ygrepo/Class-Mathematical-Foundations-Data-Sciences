\documentclass[12pt,twoside]{article}
\usepackage[dvipsnames]{xcolor}
\usepackage{tikz,graphicx,amsmath,amsfonts,amscd,amssymb,bm,cite,epsfig,epsf,url}
\usepackage[hang,flushmargin]{footmisc}
\usepackage[colorlinks=true,urlcolor=blue,citecolor=blue]{hyperref}
\usepackage{amsthm,multirow,wasysym,appendix}
\usepackage{array,subcaption} 
% \usepackage[small,bf]{caption}
\usepackage{bbm}
\usepackage{pgfplots}
\usetikzlibrary{spy}
\usepgfplotslibrary{external}
\usepgfplotslibrary{fillbetween}
\usetikzlibrary{arrows,automata}
\usepackage{thmtools}
\usepackage{blkarray} 
\usepackage{textcomp}
\usepackage[left=0.8in,right=1.0in,top=1.0in,bottom=1.0in]{geometry}

\usepackage{times}
\usepackage{amsfonts}
\usepackage{amsmath}
%\usepackage[psamsfonts]{amssymb}
\usepackage{latexsym}
\usepackage{color}
\usepackage{graphics}
\usepackage{enumerate}
\usepackage{amstext}
\usepackage{blkarray}
\usepackage{url}
\usepackage{epsfig}
\usepackage{bm}
\usepackage{hyperref}
\hypersetup{
    colorlinks=true,
    linkcolor=blue,
    filecolor=magenta,      
    urlcolor=blue,
}
\usepackage{mathtools}
\usepackage{minted}
\usepackage{bm}
%% Probability operators and functions
%
% \def \P{\mathrm{P}}
%\input{Symbols.tex}
\def \P{\mathrm{P}}
\def \E{\mathrm{E}}
\def \Var{\mathrm{Var}}
\let\var\Var
\def \Cov {\mathrm{Cov}} \let\cov\Cov
\def \MSE {\mathrm{MSE}} \let\mse\MSE
\def \sgn {\mathrm{sgn}}
\def \R {\mathbb{R}}
\def \RR {\mathbb{R}}
\def \C {\mathbb{C}}
\def \CC {\mathbb{C}}
\def \N {\mathbb{N}}
\def \Z {\mathbb{Z}}
\def \cV {\mathcal{V}}
\def \cS {\mathcal{S}}
\DeclareMathOperator{\rank}{rank}
\DeclareMathOperator*{\argmin}{arg\,min}
\DeclareMathOperator*{\argmax}{arg\,max}
\newcommand{\red}[1]{\textcolor{red}{#1}}
\newcommand{\blue}[1]{\textcolor{blue}{#1}}
\newcommand{\green}[1]{\textcolor{ForestGreen}{ #1}}
\newcommand{\fuchsia}[1]{\textcolor{RoyalPurple}{ #1}}

%
%% Probability distributions
%
%\def \Bern    {\mathrm{Bern}}
%\def \Binom   {\mathrm{Binom}}
%\def \Exp     {\mathrm{Exp}}
%\def \Geom    {\mathrm{Geom}}
\def \Norm    {\mathcal{N}}
%\def \Poisson {\mathrm{Poisson}}
%\def \Unif    {\mathrm {U}}
%
\newcommand{\bdb}[1]{\textcolor{red}{#1}}

\newcommand{\ml}[1]{\mathcal{ #1 } }
\newcommand{\wh}[1]{\widehat{ #1 } }
\newcommand{\wt}[1]{\widetilde{ #1 } }
\newcommand{\conj}[1]{\overline{ #1 } }
\newcommand{\rnd}[1]{\tilde{ #1 } }
\newcommand{\rv}[1]{ \rnd{ #1}  }
\newcommand{\rvw}{\rv{w}  }
\newcommand{\rvx}{\rv{x}  }
\newcommand{\rvy}{\rv{y}  }
\newcommand{\rvz}{\rv{z}  }
\newcommand{\rvbeta}{\rv{\beta}  }
\newcommand{\rvu}{\rv{u}  }
\newcommand{\rx}{\rnd{ x}  }
\newcommand{\ry}{\rnd{ y}  }
\newcommand{\ra}{\rnd{ a}  }
\newcommand{\rb}{\rnd{ b}  }

\def \cnd {\, | \,}
\def \Id { I }

\newcommand{\op}[1]{\operatorname{#1}}
\newcommand{\setdef}[2]{ := \keys{ #1 \; | \; #2 } }
\newcommand{\set}[2]{ \keys{ #1 \; | \; #2 } }
\newcommand{\sign}[1]{\op{sign}\left( #1 \right) }
\newcommand{\trace}[1]{\op{tr}\left( #1 \right) }
\newcommand{\tr}[1]{\op{tr}\left( #1 \right) }
\newcommand{\inv}[1]{\left( #1 \right)^{-1} }
\newcommand{\abs}[1]{\left| #1 \right|}
\newcommand{\keys}[1]{\left\{ #1 \right\}}
\newcommand{\sqbr}[1]{\left[ #1 \right]}
\newcommand{\sbrac}[1]{ ( #1 ) }
\newcommand{\brac}[1]{\left( #1 \right) }
\newcommand{\bbrac}[1]{\big( #1 \big) }
\newcommand{\Bbrac}[1]{\Big( #1 \Big)}
\newcommand{\BBbrac}[1]{\BIG( #1 \Big)}
\newcommand{\MAT}[1]{\begin{bmatrix} #1 \end{bmatrix}}
\newcommand{\sMAT}[1]{\left(\begin{smallmatrix} #1 \end{smallmatrix}\right)}
\newcommand{\sMATn}[1]{\begin{smallmatrix} #1 \end{smallmatrix}}
\newcommand{\PROD}[2]{\left \langle #1, #2\right \rangle}
\newcommand{\PRODs}[2]{\langle #1, #2 \rangle}
\newcommand{\der}[2]{\frac{\text{d}#2}{\text{d}#1}}
\newcommand{\pder}[2]{\frac{\partial#2}{\partial#1}}
\newcommand{\derTwo}[2]{\frac{\text{d}^2#2}{\text{d}#1^2}}
\newcommand{\ceil}[1]{\lceil #1 \rceil}
\newcommand{\Imag}[1]{\op{Im}\brac{ #1 }}
\newcommand{\Real}[1]{\op{Re}\brac{ #1 }}
\newcommand{\norm}[1]{\left|\left| #1 \right|\right| }
\newcommand{\normProd}[1]{\left|\left| #1 \right|\right| _{\PROD{\cdot}{\cdot}} }
\newcommand{\normTwo}[1]{\left|\left| #1 \right|\right| _{2} }
\newcommand{\normTwos}[1]{ \| #1  \| _{2} }
\newcommand{\normZero}[1]{\left|\left| #1 \right|\right| _{0} }
\newcommand{\normTV}[1]{\left|\left| #1 \right|\right|  _{ \op{TV}  } }% _{\op{c} \ell_1} }
\newcommand{\normOne}[1]{\left|\left| #1 \right|\right| _{1} }
\newcommand{\normOneTwo}[1]{\left|\left| #1 \right|\right| _{1,2} }
\newcommand{\normF}[1]{\left|\left| #1 \right|\right| _{\op{F}} }
\newcommand{\normLTwo}[1]{\left|\left| #1 \right|\right| _{\ml{L}_2} }
\newcommand{\normNuc}[1]{\left|\left| #1 \right|\right| _{\ast} }
\newcommand{\normOp}[1]{\left|\left| #1 \right|\right|  }
\newcommand{\normInf}[1]{\left|\left| #1 \right|\right| _{\infty}  }
\newcommand{\proj}[1]{\mathcal{P}_{#1} \, }
\newcommand{\diff}[1]{ \, \text{d}#1 }
\newcommand{\vc}[1]{\boldsymbol{\vec{#1}}}
\newcommand{\rc}[1]{\boldsymbol{#1}}
\newcommand{\vx}{\vec{x}}
\newcommand{\vy}{\vec{y}}
\newcommand{\vz}{\vec{z}}
\newcommand{\vu}{\vec{u}}
\newcommand{\vv}{\vec{v}}
\newcommand{\ve}{\vec{e}}
\newcommand{\valpha}{\vec{\alpha}}
\newcommand{\vbeta}{\vec{\beta}}
\newcommand{\vgamma}{\vec{\gamma}}
\newcommand{\vb}{\vec{\beta}}
\newcommand{\va}{\vec{\alpha}}
\newcommand{\vaa}{\vec{a}}
\newcommand{\vbb}{\vec{b}}
\newcommand{\vg}{\vec{g}}
\newcommand{\vw}{\vec{w}}
\newcommand{\vh}{\vec{h}}
\newcommand{\vnu}{\vec{\nu}}
\newcommand{\rvnu}{\vc{\nu}}

\newtheorem{theorem}{Theorem}[section]
% \declaretheorem[style=plain,qed=$\square$]{theorem}
\newtheorem{corollary}[theorem]{Corollary}
\newtheorem{definition}[theorem]{Definition}
\newtheorem{lemma}[theorem]{Lemma}
\newtheorem{remark}[theorem]{Remark}
\newtheorem{algorithm}[theorem]{Algorithm}


%
%% Typographic tweaks and miscellaneous
%\newcommand{\sfrac}[2]{\mbox{\small$\displaystyle\frac{#1}{#2}$}}
%\newcommand{\suchthat}{\kern0.1em{:}\kern0.3em}
%\newcommand{\qqquad}{\kern3em}
%\newcommand{\cond}{\,|\,}
%\def\Matlab{\textsc{Matlab}}
%\newcommand{\displayskip}[1]{\abovedisplayskip #1\belowdisplayskip #1}
%\newcommand{\term}[1]{\emph{#1}}
%\renewcommand{\implies}{\;\Rightarrow\;}



\begin{document}

\noindent DS-GA.1013 Mathematical Tools for Data Science \\
Homework 2 \\
Yves Greatti - yg390\\

\begin{enumerate}
\item (Correlation coefficient) The entries of a two-dimensional random vector have a correlation coefficient equal to one. What is the variance of the second principal component? Provide both a proof and an intuitive justification. 

Let X and Y the two components of a random vector. If $\rho_{X,Y} = \frac {\Cov(X,Y)}{\sigma_X \sigma_Y} = 1$ then $\Cov(X,Y) = \sigma_X \sigma_Y$
The covariance matrix  is then  $ \Sigma = \begin{bmatrix} \sigma_X^2 & \Cov(X,Y) \\  \Cov(X,Y) & \sigma_Y^2  \end{bmatrix}  = \begin{bmatrix}  \sigma_X^2 &  \sigma_X \sigma_Y \\ \sigma_X \sigma_Y & \sigma_Y^2 \end{bmatrix}$.
$\det(\Sigma -\lambda I) = \lambda (\lambda - (\sigma_X^2 +  \sigma_Y^2)), \lambda \in   \mathbf{R} \Rightarrow $ the eigenvalues are $0$ and $\sigma_X^2 +  \sigma_Y^2$.
The sample variance of the second principal component is the smaller eigenvalue $0$. We can expect such result since there $X$ and $Y$ are positively correlated with a coefficient of one thus the variance of the data is completely expressed by the variance of the first principal component, there is no information on the second principal component.


\newpage
\item (Not centering) To analyze what happens if we apply PCA without centering, let $\rx$ be a $d$-dimensional vector with mean $\mu \in \R^{d}$ and covariance matrix $\Sigma_{\rx}$ equal to the identity matrix. If we compute the eigendecomposition of the matrix $\E(\rx \rx^T)$ what is the value of the largest eigenvalue? What is the direction of the corresponding eigenvector? 
  
  \begin{align*}
  	\E((\rx - \mu) (\rx - \mu)^T))		&=	\E(\rx \rx^T -2 \mu^T \rx + \mu^T \mu) \\
								&= 	\E(\rx \rx^T) -2 \mu \E(\rx^T) + \mu \mu^T \\
								&=	\E(\rx \rx^T) -2  \mu \mu^T +  \mu \mu^T \\
								&=	\E(\rx \rx^T) - \mu \mu^T \\
	\Rightarrow	\E(\rx \rx^T) 		&=	\E((\rx - \mu) (\rx - \mu)^T)	) +  \mu \mu^T \\	
								&=	\Sigma_{\rx} +  \mu \mu^T
  \end{align*}
  
We have
  \begin{align*}
  		\E(\rx \rx^T) =
	   \begin{bmatrix}
	   		1+ \mu_1^2	& 	\mu_1 \mu_2	&	\ldots	& 	\mu_1 \mu_d	\\
			\mu_1 \mu_2	&	1+ \mu_2^2	&	\ldots	& 	\mu_2 \mu_d	\\
			\vdots		&	\ddots		&			& 	\vdots		\\
			\mu_1 \mu_d	&	\ldots		&			&	1 + \mu_d^2	
	   \end{bmatrix}
  \end{align*}
  We note that an eigenvectors of this matrix is
   $u = \begin{bmatrix}
  	\mu_1	& \mu_2	& \ldots	& \mu_d
  \end{bmatrix}^T$ since $\E(\rx \rx^T) u = (1 + \sum_{i=1}^d \mu_i^2) u$.
Let's prove that if $A$ and $B=I+A$ matrices share the same number of eigenvalues and eigenvector.
If $\lambda$ and the corresponding eigenvector $v$, $v \neq 0$,  is an eigenvalue of $A$ then $B v = (I+A) v = (1+\lambda) v$ thus 
$(1+\lambda)$ is an eigenvalue of $B$ and v is also an eigenvector. Now if $\mu$ is an eigenvalue of $B$ then $ A v = (A+I) v - I v = B v - I v = (\mu -1) v$ thus
$(\mu -1)$ is an eigenvalue of $A$ for the same eigenvector $v$.  Let $A=
	   \begin{bmatrix}
	   		\mu_1^2	& 	\mu_1 \mu_2	&	\ldots	& 	\mu_1 \mu_d	\\
			\mu_1 \mu_2	&	\mu_2^2	&	\ldots	& 	\mu_2 \mu_d	\\
			\vdots		&	\ddots		&			& 	\vdots		\\
			\mu_1 \mu_d	&	\ldots		&			&	\mu_d^2	
	   \end{bmatrix}
$, we know that $\sum_{i=1}^d \lambda_i = \trace{A} = \sum_{i=1}^d \mu_i^2$ where $\lambda_i, i=1, \dots, d$  are the eigenvalues of $A$
 thus  $(\sum_{i=1}^d \mu_i^2)$ is the only eigenvalue $A$.

Setting $\E(\rx \rx^T)=B = I + A$, we can conclude that the largest eigenvalue of $\E(\rx \rx^T)$ is $1 + \|\mu\|_2^2$ with eigenvector 
 $u = \begin{bmatrix}
  	\mu_1	& \mu_2	& \ldots	& \mu_d
  \end{bmatrix}^T$.
  
 \newpage
  
 \item (Financial data) In this exercise you will use the code in the findata folder.
  For the data loading code to work properly, make sure you
  have the pandas Python package installed on your system.

  Throughout, we will be using the data obtained by calling
 \emph{load\_data} in \emph{findata\_tools.py}.  This will
  give you the names, and closing prices for a set of 18 stocks over a
  period of 433 days ordered chronologically.
  For a fixed stock (such as msft), let
  $P_1,\ldots,P_{433}$ denote its sequence of closing prices ordered in
  time.  For that stock, define the daily returns series $R_i:=P_{i+1}-P_i$ for
  $i=1,\ldots,432$.  Throughout we think of the daily stock returns as features,
  and each day (but the last) as a separate datapoint in $\R^{18}$.
  That is, we have $432$ datapoints each having $18$ features.
  \begin{enumerate}
  \item Looking at the first two principal directions of the
    centered data, give the two stocks with the largest
    coefficients (in absolute value) in each direction.  
    Give a hypothesis why these two stocks have the largest
    coefficients, and confirm your hypothesis using the data.  The file 
   \emph{findata\_tools.py} has\emph{pretty\_print}
    functions that can help you output your results.
    You are not required to include the principal directions in
    your submission.\\ \\
    The two stocks corresponding to the two principal directions of the centered data with the largest coefficients (in absolute value) in each direction
    are: "amzn" and  "goog". It can be explained by computing the absolute return for each stock over the period of 433 days:

	\begin{figure}[H]
		\centering
		\includegraphics[width=200pt]{figures/pb_3_a.png}
		\caption{Stocks returns over a period of 433 days (output of pretty\_print).}
		\label{fig1}
	\end{figure}
    
    In term of return goog and amzn stocks returned about 4 and  3 times more than the next stock after amzn and goog stocks with the highest return: gs, 53 and 43 times more than
    the last stock in term of return among the 18 stocks: xlf.
    \begin{center}
    		\begin{tabular}{ | c | c | c | c | }
    		\hline
			\text{amzn/gs} & \text{amzn/xlf} & \text{goog/gs}  & \text{goog/xlf} \\
		\hline
			3.8 & 53.2 & 3.1 & 43.5  \\ 
		\hline
    	\end{tabular}
    \end{center}
   So most of the variance in the data will be explained by these two stocks: goog and amzn.

  \item Standardize the centered data so that each stock (feature) has
    variance 1 and compute the first 2 principal directions.  This is
    equivalent to computing the principal directions of the
    correlation matrix (the previous part used the covariance
    matrix).  Using the information in the comments of
   \emph{generate\_findata.py} as a guide to the stocks, 
    give an English interpretation of the first 2 principal directions
    computed here. 
    You are not required to include the principal directions in
    your submission.
     
    We can think of each of the entries of the principal directions as a weighting on the corresponding stock.  
    	\bi
    		\item SPY - A security that roughly tracks the S\&P 500, a weighted average of the stock prices of 500 top US companies.
		We have only 18 stocks and excluding the ETF (Exchange traded products), we are left with 13 stocks. 
		We can notice that among these 13 stocks, SPY entry in the principal directions, has the same than the other 13 entries
		for the same principal direction.
		For a given day, computing a weighted average using the absolute values of entries in the first principal direction and the prices of these 13 stocks, 
		we obtain a weighted average price in the range of the SPY price reported for the same day. It is less true using the entries of the second principal direction.

		\begin{figure}[H]
			\centering
			\includegraphics[width=200pt]{figures/pb_3_b_principal_directions.png}
			\caption{First two Principal Directions.}
			\label{fig3}
		\end{figure}

		\begin{figure}[H]
			\centering
			\includegraphics[width=200pt]{figures/pb_3_b_1.png}
			\caption{Weighted average of the 13 stock prices and SPY price using PD0.}
			\label{fig2}
		\end{figure}
					
		\item  XLF -  A security that tracks a weighted average of top US financial companies.
		Among the 18 stocks, there are only 3 financial stocks: c, gs and jpm. Entry for xlf has the same sign as the entries for these three stocks in the first two 
		principal directions. To some extent the weighted average price of these 3 stocks is somewhat close to XLF price.
		
		\begin{figure}[H]
			\centering
			\includegraphics[width=200pt]{figures/pb_3_b_3.png}
			\caption{Weighted average of the 3 financial stock prices and XLF price using PD0.}
			\label{fig4}
		\end{figure}
	
		\item SSO - ProShares levered ETF that roughly corresponds to twice the daily performance of the S\&P 500.
		We have similar entries for SSO and SPY which indicate a strong correlation between these two stocks
		\begin{center}
    			\begin{tabular}{ | c | c | c | }
    			\hline
								& \text{SSO} 	& \text{SPY} \\
			\hline
				\text{First PD}		& -0.3348		& -0.3366  \\ 
    			\hline
				\text{Second PD}	& 0.09156		& 0.0848  \\ 
			\hline
    			\end{tabular}
   		 \end{center}
		 We can also pick two random days (200 and 201) and compare the returns for spy vs. xlf, you can see the return for xlf tracks the return of spy:
		 \begin{figure}[H]
			\centering
			\includegraphics[width=200pt]{figures/sso_200.png}
			\caption{Returns for day 200}
		 \end{figure}
		 \begin{figure}[H]
			\centering
			\includegraphics[width=200pt]{figures/sso_201.png}
			\caption{Returns for day 201}
		 \end{figure}

		 
		 
		\item SDS - ProShares inverse levered ETF that roughly corresponds to twice the negative daily performance of the S\&P 500. 
		The entries for SDS and SPY are roughly opposite confirming the opposite trend of SDS compared to SPY.
		\begin{center}
    			\begin{tabular}{ | c | c | c | }
    			\hline
								& \text{SDS} 	& \text{SPY} \\
			\hline
				\text{First PD}		& 0.3272		& -0.3366  \\ 
    			\hline
				\text{Second PD}	& -0.0560		& 0.0848  \\ 
			\hline
    			\end{tabular}
   		 \end{center}
		 
		\item USO - Exchange traded product that tracks the price of oil in the US
		Taking the mean of the entries related to oil company (xom, apc, cvx) from the principal directions  and comparing to the entry for USO, they are close,
		confirming the correlation between uso and (xom, apc, cvx):
		\begin{center}
    			\begin{tabular}{ | c | c | c | }
    			\hline
								& \text{USO} 	& \text{Mean(XOM, APC, CVX)} \\
			\hline
				\text{First PD}		& -0.1592		& -0.2091 \\ 
    			\hline
				\text{Second PD}	& -0.3709		& -0.3102  \\ 
			\hline
    			\end{tabular}
   		 \end{center}
		

   	\ei
     
    
  \item Assume the stock returns each day are drawn independently from a
    multivariate distribution $\rx$ where
    $\rx[i]$ corresponds to the $i$th stock.  Assume further that
    you hold a portfolio with $200$ shares of each of appl, amzn, msft, and
    goog, and $100$ shares of each of the remaining 14 stocks in the
    dataset.  Using the sample covariance matrix as an estimator for
    the true covariance of $\rx$, approximate the standard deviation of
    your 1 day portfolio returns $\ry$ (this is a measure of the risk of your
    portfolio).  Here $\ry$ is given by
    $$\ry := \sum_{i=1}^{18} \alpha[i] \rx[i],$$
    where $\alpha[i]$ is the number of shares you hold of stock $i$.  
    
    Using the sample covariance matrix and taking the root square of [$ \text{shares}^T \times \text{covariance} \times \text{shares}$], we find that for such portfolio the standard deviation of 1 day is: $4309.94952$.
    
  \item Assume further that $\rx$ from the previous part has a
    multivariate Gaussian distribution.  Compute the probability
    of losing $1000$ or more dollars in a single day.  That is,
    compute
    $$\Pr(\ry \leq -1000).$$
    For each day of the $432$ days, we compute the daily return $\ry := \sum_{i=1}^{18} \alpha[i] \rx[i]$ and count the number of times over the $432$ days: $\Pr(\ry \leq -1000)$, then divide the result by the number of days (432),
    we obtain: $0.3425$.
    
  \end{enumerate}
  Note: The assumptions made in the previous parts are often
  invalid and can lead to inaccurate risk calculations in real
  financial situations. 
  
  \newpage
  \item The following questions refer to the code in the folder {\tt faces }
  The Olivetti faces dataset used in  {\tt faces }  contains images of faces of people associated with a unique numeric id to identify the person.

  	\begin{enumerate}
		\item Complete the \verb|compute_nearest_neighbors()| function in \verb|nearest_neighbors.py| that finds the image in the training data that is closest to a given test image.  Include the generated images in your submitted homework.


		The data set consists in $400$ rows of a ravelled face image of original size 64 x 64 pixels . Each row represent a datapoint in $\R^{4096}$.
		A label is associated to each face image which correspond to the Subject IDs.
		In the \emph{nearest\_neighbors} function we compute the distance between a test image and a set of reference image (train\_matrix).

\begin{minted}{python}
	def compute_nearest_neighbors(train_matrix, testImage):
	   distances = np.sqrt(
	   	np.sum((train_matrix - testImage) ** 2, 
	   			axis=1))
    	    idx_of_closest_point_in_train_matrix = 
	    	np.argsort(distances)
	    return idx_of_closest_point_in_train_matrix[0]
\end{minted}


	\begin{figure}[H]
		\centering
		\includegraphics[width=200pt]{figures/Image-NearestNeighbor}
		\caption{Test images and image found by Nearest Neighbors.}
		\label{fig5}
	\end{figure}


\vspace{0.5cm}Create a new file  in which
you must write code to complete the following tasks:


\item Generate a plot of $k$ vs.~$\sigma^2_k$, where $\sigma^2_k$ is
the variance  with the $k$th principal component of the data (e.g., $\sigma^2_1$
is the largest variance). 
Include
the plot in your submitted homework document. You can limit the x axis to a reasonable number.

We use PCA decomposition from sklearn which gives us the singular values ordered which correspond to the variance with each principal component, we select $k$ of them
and plot them for each principal component:
\begin{minted}{python}
faces = fetch_olivetti_faces().data
n_samples, n_features = faces.shape
faces_centered = faces - faces.mean(axis=0)
cov = np.cov(faces_centered, rowvar=False)
eigvals, _ = np.linalg.eigh(cov)

k = 40
truncated_eigvals = eigvals[::-1][:k]
fig, ax = plt.subplots(figsize=(10, 6))
k_range = range(1, k+1)
label_str = "variance for largest {} 
principal components".format(k)
ax.plot(k_range, truncated_eigvals, "-", color="red", 
label=label_str)
ax.set_xlabel("principal component")
ax.set_ylabel("Variance")
ax.set_title(r"Explained Variance of the $k^{th}$ component")
ax.legend()
plt.show();
fig.savefig("pb_4_b.pdf",bbox_inches='tight');
\end{minted}

	\begin{figure}[H]
		\centering
		\includegraphics[width=200pt]{figures/pb_4_b.pdf}
		\caption{Variance with principal components}
		\label{fig6}
	\end{figure}

\item  Plot (using  \verb|plot_image_grid()| in
\verb|plot_tools.py| ) the vectors
corresponding to the top 10 principal directions of the data.
Your principal direction vectors should be elements of $\mathbb{R}^{4096}$ (i.e., they
should represent images).
Include the plot in your submitted homework document.

We compute the sample covariance matrix of the data $\Sigma_{\mathcal{X}}$ ,
 the principal directions are the eigenvectors of the eigendecomposition of  $\Sigma_{\mathcal{X}}$:
 
\begin{minted}{python} 
faces = fetch_olivetti_faces().data
n_samples, n_features = faces.shape
faces_centered = faces - faces.mean(axis=0)
cov = np.cov(faces_centered, rowvar=False)
_, principal_directions = np.linalg.eigh(cov)
k = 10
top_pd = principal_directions[:, ::-1][:,:k].T
title = "Top {} principal directions vectors".format(k)
plot_tools.plot_image_grid(top_pd, title)
\end{minted}

	\begin{figure}[H]
		\centering
		\includegraphics[width=200pt]{figures/Top_10_principal_direction_vectors.pdf}
		\caption{Top 10 principal direction vectors}
		\label{fig7}
	\end{figure}


\item Use the variance of principal directions plot to determine a 
realtively small number $k$ of principal components
that explains the training data reasonably well.
Project the training data 
and the test data onto the
first $k$ principal components, and run nearest neighbors for
each test image in this lower dimensional space.  Include your
choice for $k$, and the plots
of your nearest neighbor results in your submitted homework
document.  You should use the code from
 \verb|nearest_neighbors.py| to generate your image plots.
Based on the variance plot, most of the variance of the data is captured by the $40$ first principal components. Using the first 40 principal components, we project the training
and test data set onto these principal components before finding the nearest neighbors.

\begin{minted}{python} 

def compute_nearest_neighbors(train_matrix, testImage):
	   distances = np.sqrt(
	   	np.sum((train_matrix - testImage) ** 2, 
	   			axis=1))
    	    idx_of_closest_point_in_train_matrix = 
	    	np.argsort(distances)
	    return idx_of_closest_point_in_train_matrix[0]

test_idx = [1, 87, 94, 78]

faces = fetch_olivetti_faces()
targets = faces.target
faces = faces.images.reshape((len(faces.images), -1))

train_idx = np.array(list(set(list(range(faces.shape[0]))) 
- set(test_idx)))

train_set = faces[train_idx]
y_train = targets[train_idx]
test_set = faces[np.array(test_idx)]
y_test = targets[np.array(test_idx)]

k = 40
model = PCA(n_components=k)
# Do PCA and compute principal directions.
model.fit(faces);


top_principal_components = model.components_

# Projection of training and test data
projected_train_set =
 train_set.dot(top_principal_components.T)
projected_test_set = 
test_set.dot(top_principal_components.T)

imgs = list()
est_labels = list()
for i in range(projected_test_set.shape[0]):
    test_image = projected_test_set[i, :]
    # Nearest neighbors in smaller dimension space
    nnIdx = compute_nearest_neighbors(projected_train_set, 
    test_image)
    imgs.extend([test_set[i,:], train_set[nnIdx, :]])
    est_labels.append(y_train[nnIdx])

row_titles = ['Test', 'Nearest']
col_titles = 
['%d vs. %d' % (i, j) for i, j in zip(y_test, est_labels)]
plot_tools.plot_image_grid(imgs,
                               "PC-Image-NearestNeighbor",
                               (64, 64), len(projected_test_set), 
                           n_row=2, bycol=True, row_titles=row_titles,
                            col_titles=col_titles)
\end{minted}

	\begin{figure}[H]
		\centering
		\includegraphics[width=200pt]{figures/PC-Image-NearestNeighbor.pdf}
		\caption{Nearest neighbors using top 10 principal component vectors}
		\label{fig8}
	\end{figure}


\item Give a potential reason why the principal component-based 
nearest-neighbor approach used in the previous part could be
more accurate than using the full training set.

Using the top 40 principal directions, we keep most of the variance of the data along these directions and by projecting into the span of these vectors, we can expect to have capture most of the significant characteristics of the data in a space of smaller dimensions ($\mathbb{R}^{40}$ vs. $\mathbb{R}^{4096}$). So performing nearest neighbors in the space with smaller dimension could be more or as accurate than doing it in the original space.
 
\item Use \verb|sklearn.cluster.KMeans| to perform KMeans on the entire dataset (both train and test set) with $k=40$. Use \verb|plot_image_grid()| to create a picture of all the $k$ cluster centers. 

\begin{minted}{python} 
rng = RandomState(0)
center = True
n_components = 40

faces = fetch_olivetti_faces().data
faces_centered = faces - faces.mean(axis=0)
estimator =  KMeans(n_clusters=n_components, random_state=rng)
estimator.fit(faces)
kmeans_components = estimator.cluster_centers_
plot_tools.plot_image_grid(kmeans_components,
 "KMEANS_clusters", (64, 64), 10, n_row=4, bycol=True)
\end{minted}

	\begin{figure}[H]
		\centering
		\includegraphics[width=200pt]{figures/KMEANS_clusters}
		\caption{40 KMeans clusters}
		\label{fig9}
	\end{figure}
	


 Some notes to keep in mind:
  \begin{enumerate}
  \item The function {\tt np.linalg.eig} might return complex eigenvectors.
  \item The data points in the training and test data are given as
    rows.
    \item Include all new code (or functions) you have filled in your final PDF.
 \end{enumerate}
 


	\end{enumerate}
  
\end{enumerate}
\end{document}
