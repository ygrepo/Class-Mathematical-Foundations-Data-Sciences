\documentclass[12pt,twoside]{article}
\usepackage[dvipsnames]{xcolor}
\usepackage{tikz,graphicx,amsmath,amsfonts,amscd,amssymb,bm,cite,epsfig,epsf,url}
\usepackage[hang,flushmargin]{footmisc}
\usepackage[colorlinks=true,urlcolor=blue,citecolor=blue]{hyperref}
\usepackage{amsthm,multirow,wasysym,appendix}
\usepackage{array,subcaption} 
% \usepackage[small,bf]{caption}
\usepackage{bbm}
\usepackage{pgfplots}
\usetikzlibrary{spy}
\usepgfplotslibrary{external}
\usepgfplotslibrary{fillbetween}
\usetikzlibrary{arrows,automata}
\usepackage{thmtools}
\usepackage{blkarray} 
\usepackage{textcomp}
\usepackage[left=0.8in,right=1.0in,top=1.0in,bottom=1.0in]{geometry}

\usepackage{bm}
%% Probability operators and functions
%
% \def \P{\mathrm{P}}
%\input{Symbols.tex}
\def \P{\mathrm{P}}
\def \E{\mathrm{E}}
\def \Var{\mathrm{Var}}
\let\var\Var
\def \Cov {\mathrm{Cov}} \let\cov\Cov
\def \MSE {\mathrm{MSE}} \let\mse\MSE
\def \sgn {\mathrm{sgn}}
\def \R {\mathbb{R}}
\def \RR {\mathbb{R}}
\def \C {\mathbb{C}}
\def \CC {\mathbb{C}}
\def \N {\mathbb{N}}
\def \Z {\mathbb{Z}}
\def \cV {\mathcal{V}}
\def \cS {\mathcal{S}}
\DeclareMathOperator{\rank}{rank}
\DeclareMathOperator*{\argmin}{arg\,min}
\DeclareMathOperator*{\argmax}{arg\,max}
\newcommand{\red}[1]{\textcolor{red}{#1}}
\newcommand{\blue}[1]{\textcolor{blue}{#1}}
\newcommand{\green}[1]{\textcolor{ForestGreen}{ #1}}
\newcommand{\fuchsia}[1]{\textcolor{RoyalPurple}{ #1}}

%
%% Probability distributions
%
%\def \Bern    {\mathrm{Bern}}
%\def \Binom   {\mathrm{Binom}}
%\def \Exp     {\mathrm{Exp}}
%\def \Geom    {\mathrm{Geom}}
\def \Norm    {\mathcal{N}}
%\def \Poisson {\mathrm{Poisson}}
%\def \Unif    {\mathrm {U}}
%
\newcommand{\bdb}[1]{\textcolor{red}{#1}}

\newcommand{\ml}[1]{\mathcal{ #1 } }
\newcommand{\wh}[1]{\widehat{ #1 } }
\newcommand{\wt}[1]{\widetilde{ #1 } }
\newcommand{\conj}[1]{\overline{ #1 } }
\newcommand{\rnd}[1]{\tilde{ #1 } }
\newcommand{\rv}[1]{ \rnd{ #1}  }
\newcommand{\rvw}{\rv{w}  }
\newcommand{\rvx}{\rv{x}  }
\newcommand{\rvy}{\rv{y}  }
\newcommand{\rvz}{\rv{z}  }
\newcommand{\rvbeta}{\rv{\beta}  }
\newcommand{\rvu}{\rv{u}  }
\newcommand{\rx}{\rnd{ x}  }
\newcommand{\ry}{\rnd{ y}  }
\newcommand{\ra}{\rnd{ a}  }
\newcommand{\rb}{\rnd{ b}  }

\def \cnd {\, | \,}
\def \Id { I }

\newcommand{\op}[1]{\operatorname{#1}}
\newcommand{\setdef}[2]{ := \keys{ #1 \; | \; #2 } }
\newcommand{\set}[2]{ \keys{ #1 \; | \; #2 } }
\newcommand{\sign}[1]{\op{sign}\left( #1 \right) }
\newcommand{\trace}[1]{\op{tr}\left( #1 \right) }
\newcommand{\tr}[1]{\op{tr}\left( #1 \right) }
\newcommand{\inv}[1]{\left( #1 \right)^{-1} }
\newcommand{\abs}[1]{\left| #1 \right|}
\newcommand{\keys}[1]{\left\{ #1 \right\}}
\newcommand{\sqbr}[1]{\left[ #1 \right]}
\newcommand{\sbrac}[1]{ ( #1 ) }
\newcommand{\brac}[1]{\left( #1 \right) }
\newcommand{\bbrac}[1]{\big( #1 \big) }
\newcommand{\Bbrac}[1]{\Big( #1 \Big)}
\newcommand{\BBbrac}[1]{\BIG( #1 \Big)}
\newcommand{\MAT}[1]{\begin{bmatrix} #1 \end{bmatrix}}
\newcommand{\sMAT}[1]{\left(\begin{smallmatrix} #1 \end{smallmatrix}\right)}
\newcommand{\sMATn}[1]{\begin{smallmatrix} #1 \end{smallmatrix}}
\newcommand{\PROD}[2]{\left \langle #1, #2\right \rangle}
\newcommand{\PRODs}[2]{\langle #1, #2 \rangle}
\newcommand{\der}[2]{\frac{\text{d}#2}{\text{d}#1}}
\newcommand{\pder}[2]{\frac{\partial#2}{\partial#1}}
\newcommand{\derTwo}[2]{\frac{\text{d}^2#2}{\text{d}#1^2}}
\newcommand{\ceil}[1]{\lceil #1 \rceil}
\newcommand{\Imag}[1]{\op{Im}\brac{ #1 }}
\newcommand{\Real}[1]{\op{Re}\brac{ #1 }}
\newcommand{\norm}[1]{\left|\left| #1 \right|\right| }
\newcommand{\normProd}[1]{\left|\left| #1 \right|\right| _{\PROD{\cdot}{\cdot}} }
\newcommand{\normTwo}[1]{\left|\left| #1 \right|\right| _{2} }
\newcommand{\normTwos}[1]{ \| #1  \| _{2} }
\newcommand{\normZero}[1]{\left|\left| #1 \right|\right| _{0} }
\newcommand{\normTV}[1]{\left|\left| #1 \right|\right|  _{ \op{TV}  } }% _{\op{c} \ell_1} }
\newcommand{\normOne}[1]{\left|\left| #1 \right|\right| _{1} }
\newcommand{\normOneTwo}[1]{\left|\left| #1 \right|\right| _{1,2} }
\newcommand{\normF}[1]{\left|\left| #1 \right|\right| _{\op{F}} }
\newcommand{\normLTwo}[1]{\left|\left| #1 \right|\right| _{\ml{L}_2} }
\newcommand{\normNuc}[1]{\left|\left| #1 \right|\right| _{\ast} }
\newcommand{\normOp}[1]{\left|\left| #1 \right|\right|  }
\newcommand{\normInf}[1]{\left|\left| #1 \right|\right| _{\infty}  }
\newcommand{\proj}[1]{\mathcal{P}_{#1} \, }
\newcommand{\diff}[1]{ \, \text{d}#1 }
\newcommand{\vc}[1]{\boldsymbol{\vec{#1}}}
\newcommand{\rc}[1]{\boldsymbol{#1}}
\newcommand{\vx}{\vec{x}}
\newcommand{\vy}{\vec{y}}
\newcommand{\vz}{\vec{z}}
\newcommand{\vu}{\vec{u}}
\newcommand{\vv}{\vec{v}}
\newcommand{\ve}{\vec{e}}
\newcommand{\valpha}{\vec{\alpha}}
\newcommand{\vbeta}{\vec{\beta}}
\newcommand{\vgamma}{\vec{\gamma}}
\newcommand{\vb}{\vec{\beta}}
\newcommand{\va}{\vec{\alpha}}
\newcommand{\vaa}{\vec{a}}
\newcommand{\vbb}{\vec{b}}
\newcommand{\vg}{\vec{g}}
\newcommand{\vw}{\vec{w}}
\newcommand{\vh}{\vec{h}}
\newcommand{\vnu}{\vec{\nu}}
\newcommand{\rvnu}{\vc{\nu}}

\newtheorem{theorem}{Theorem}[section]
% \declaretheorem[style=plain,qed=$\square$]{theorem}
\newtheorem{corollary}[theorem]{Corollary}
\newtheorem{definition}[theorem]{Definition}
\newtheorem{lemma}[theorem]{Lemma}
\newtheorem{remark}[theorem]{Remark}
\newtheorem{algorithm}[theorem]{Algorithm}


%
%% Typographic tweaks and miscellaneous
%\newcommand{\sfrac}[2]{\mbox{\small$\displaystyle\frac{#1}{#2}$}}
%\newcommand{\suchthat}{\kern0.1em{:}\kern0.3em}
%\newcommand{\qqquad}{\kern3em}
%\newcommand{\cond}{\,|\,}
%\def\Matlab{\textsc{Matlab}}
%\newcommand{\displayskip}[1]{\abovedisplayskip #1\belowdisplayskip #1}
%\newcommand{\term}[1]{\emph{#1}}
%\renewcommand{\implies}{\;\Rightarrow\;}



\begin{document}

\begin{center}
{\large{\textbf{Homework 2}} } \vspace{0.2cm}\\
Due February 23 at 11 pm
\end{center}

\begin{enumerate}
\item (Correlation coefficient) The entries of a two-dimensional random vector have a correlation coefficient equal to one. What is the variance of the second principal component? Provide both a proof and an intuitive justification. 

\item (Not centering) To analyze what happens if we apply PCA without centering, let $\rx$ be a $d$-dimensional vector with mean $\mu \in \R^{d}$ and covariance matrix $\Sigma_{\rx}$ equal to the identity matrix. If we compute the eigendecomposition of the matrix $\E(\rx \rx^T)$ what is the value of the largest eigenvalue? What is the direction of the corresponding eigenvector? 
  
 \item (Financial data) In this exercise you will use the code in the findata folder.
  For the data loading code to work properly, make sure you
  have the pandas Python package installed on your system.

  Throughout, we will be using the data obtained by calling
 \verb|load_data()| in \verb|findata_tools.py|.  This will
  give you the names, and closing prices for a set of 18 stocks over a
  period of 433 days ordered chronologically.
  For a fixed stock (such as msft), let
  $P_1,\ldots,P_{433}$ denote its sequence of closing prices ordered in
  time.  For that stock, define the daily returns series $R_i:=P_{i+1}-P_i$ for
  $i=1,\ldots,432$.  Throughout we think of the daily stock returns as features,
  and each day (but the last) as a separate datapoint in $\R^{18}$.
  That is, we have $432$ datapoints each having $18$ features.
  \begin{enumerate}
  \item Looking at the first two principal directions of the
    centered data, give the two stocks with the largest
    coefficients (in absolute value) in each direction.  
    Give a hypothesis why these two stocks have the largest
    coefficients, and confirm your hypothesis using the data.  The file 
 \verb|findata_tools.py| has \verb|pretty_print()|
    functions that can help you output your results.
    You are not required to include the principal directions in
    your submission.
  \item Standardize the centered data so that each stock (feature) has
    variance 1 and compute the first 2 principal directions.  This is
    equivalent to computing the principal directions of the
    correlation matrix (the previous part used the covariance
    matrix).  Using the information in the comments of
   \emph{generate\_findata.py} as a guide to the stocks, 
    give an English interpretation of the first 2 principal directions
    computed here. 
    You are not required to include the principal directions in
    your submission.
  \item Assume the stock returns each day are drawn independently from a
    multivariate distribution $\rx$ where
    $\rx[i]$ corresponds to the $i$th stock.  Assume further that
    you hold a portfolio with $200$ shares of each of appl, amzn, msft, and
    goog, and $100$ shares of each of the remaining 14 stocks in the
    dataset.  Using the sample covariance matrix as an estimator for
    the true covariance of $\rx$, approximate the standard deviation of
    your 1 day portfolio returns $\ry$ (this is a measure of the risk of your
    portfolio).  Here $\ry$ is given by
    $$\ry := \sum_{i=1}^{18} \alpha[i] \rx[i],$$
    where $\alpha[i]$ is the number of shares you hold of stock $i$.  
  \item Assume further that $\rx$ from the previous part has a
    multivariate Gaussian distribution.  Compute the probability
    of losing $1000$ or more dollars in a single day.  That is,
    compute
    $$\Pr(\ry \leq -1000).$$
  \end{enumerate}
  Note: The assumptions made in the previous parts are often
  invalid and can lead to inaccurate risk calculations in real
  financial situations.   


\item The following questions refer to the code in the folder {\tt faces }
  The Olivetti faces dataset used in  {\tt faces }  contains images of faces of people associated with a unique numeric id to identify the person.
 
\begin{enumerate}
\item  Complete the \verb|compute_nearest_neighbors()| function in \verb|nearest_neighbors.py| that finds the image in the training data that is closest to a given test image.  Include the generated images in your submitted homework.


\vspace{0.5cm}Create a new file  in which
you must write code to complete the following tasks:

\item Generate a plot of $k$ vs.~$\sigma^2_k$, where $\sigma^2_k$ is
the variance  with the $k$th principal component of the data (e.g., $\sigma^2_1$
is the largest variance). 
Include
the plot in your submitted homework document. You can limit the x axis to a reasonable number.
\item  Plot (using  \verb|plot_image_grid()| in
\verb|plot_tools.py| ) the vectors
corresponding to the top 10 principal directions of the data.
Your principal direction vectors should be elements of $\mathbb{R}^{4096}$ (i.e., they
should represent images).
Include the plot in your submitted homework document.
\item Use the variance of principal directions plot to determine a 
realtively small number $k$ of principal components
that explains the training data reasonably well.
Project the training data 
and the test data onto the
first $k$ principal components, and run nearest neighbors for
each test image in this lower dimensional space.  Include your
choice for $k$, and the plots
of your nearest neighbor results in your submitted homework
document.  You should use the code from
 \verb|nearest_neighbors.py| to generate your image plots.
\item Give a potential reason why the principal component-based 
nearest-neighbor approach used in the previous part could be
more accurate than using the full training set.
\item Use \verb|sklearn.cluster.KMeans| to perform KMeans on the entire dataset (both train and test set) with $k=40$. Use \verb|plot_image_grid()| to create a picture of all the $k$ cluster centers. 
\end{enumerate}

  Some notes to keep in mind:
  \begin{enumerate}
  \item The function {\tt np.linalg.eig} might return complex eigenvectors.
  \item The data points in the training and test data are given as
    rows.
    \item Include all new code (or functions) you have filled in your final PDF.
 \end{enumerate}
 
 \end{enumerate}
 
\end{document}
