\documentclass[12pt,twoside]{article}
\usepackage[dvipsnames]{xcolor}
\usepackage{tikz,graphicx,amsmath,amsfonts,amscd,amssymb,bm,cite,epsfig,epsf,url}
\usepackage[hang,flushmargin]{footmisc}
\usepackage[colorlinks=true,urlcolor=blue,citecolor=blue]{hyperref}
\usepackage{amsthm,multirow,wasysym,appendix}
\usepackage{array,subcaption} 
% \usepackage[small,bf]{caption}
\usepackage{bbm}
\usepackage{pgfplots}
\usetikzlibrary{spy}
\usepgfplotslibrary{external}
\usepgfplotslibrary{fillbetween}
\usetikzlibrary{arrows,automata}
\usepackage{thmtools}
\usepackage{blkarray} 
\usepackage{textcomp}
\usepackage[left=0.8in,right=1.0in,top=1.0in,bottom=1.0in]{geometry}

%% Probability operators and functions
%
% \def \P{\mathrm{P}}
\def \P{\mathrm{P}}
\def \E{\mathrm{E}}
\def \Var{\mathrm{Var}}
\let\var\Var
\def \Cov {\mathrm{Cov}} \let\cov\Cov
\def \MSE {\mathrm{MSE}} \let\mse\MSE
\def \sgn {\mathrm{sgn}}
\def \R {\mathbb{R}}
\def \C {\mathbb{C}}
\def \N {\mathbb{N}}
\def \Z {\mathbb{Z}}
\def \cV {\mathcal{V}}
\def \cS {\mathcal{S}}
\DeclareMathOperator*{\argmin}{arg\,min}
\DeclareMathOperator*{\argmax}{arg\,max}
\newcommand{\red}[1]{\textcolor{red}{#1}}
\newcommand{\blue}[1]{\textcolor{blue}{#1}}
\newcommand{\green}[1]{\textcolor{ForestGreen}{ #1}}
\newcommand{\fuchsia}[1]{\textcolor{RoyalPurple}{ #1}}

%
%% Probability distributions
%
%\def \Bern    {\mathrm{Bern}}
%\def \Binom   {\mathrm{Binom}}
%\def \Exp     {\mathrm{Exp}}
%\def \Geom    {\mathrm{Geom}}
%\def \Norm    {\mathcal{N}}
%\def \Poisson {\mathrm{Poisson}}
%\def \Unif    {\mathrm {U}}
%
\newcommand{\bdb}[1]{\textcolor{red}{#1}}

\newcommand{\ml}[1]{\mathcal{ #1 } }
\newcommand{\wh}[1]{\widehat{ #1 } }
\newcommand{\wt}[1]{\widetilde{ #1 } }
\newcommand{\conj}[1]{\overline{ #1 } }
\newcommand{\rnd}[1]{\tilde{ #1 } }
\newcommand{\rv}[1]{ \rnd{ #1}  }
\newcommand{\rx}{\rnd{ x}  }
\newcommand{\ry}{\rnd{ y}  }
\newcommand{\ra}{\rnd{ a}  }
\newcommand{\rb}{\rnd{ b}  }
\newcommand{\rpc}{\widetilde{ pc}  }

\def \cnd {\, | \,}
\def \Id { I }
\def \J {\mathbf{1}\mathbf{1}^T}

\newcommand{\op}[1]{\operatorname{#1}}
\newcommand{\setdef}[2]{ := \keys{ #1 \; | \; #2 } }
\newcommand{\set}[2]{ \keys{ #1 \; | \; #2 } }
\newcommand{\sign}[1]{\op{sign}\left( #1 \right) }
\newcommand{\trace}[1]{\op{tr}\left( #1 \right) }
\newcommand{\tr}[1]{\op{tr}\left( #1 \right) }
\newcommand{\inv}[1]{\left( #1 \right)^{-1} }
\newcommand{\abs}[1]{\left| #1 \right|}
\newcommand{\sabs}[1]{| #1 |}
\newcommand{\keys}[1]{\left\{ #1 \right\}}
\newcommand{\sqbr}[1]{\left[ #1 \right]}
\newcommand{\sbrac}[1]{ ( #1 ) }
\newcommand{\brac}[1]{\left( #1 \right) }
\newcommand{\bbrac}[1]{\big( #1 \big) }
\newcommand{\Bbrac}[1]{\Big( #1 \Big)}
\newcommand{\BBbrac}[1]{\BIG( #1 \Big)}
\newcommand{\MAT}[1]{\begin{bmatrix} #1 \end{bmatrix}}
\newcommand{\sMAT}[1]{\left(\begin{smallmatrix} #1 \end{smallmatrix}\right)}
\newcommand{\sMATn}[1]{\begin{smallmatrix} #1 \end{smallmatrix}}
\newcommand{\PROD}[2]{\left \langle #1, #2\right \rangle}
\newcommand{\PRODs}[2]{\langle #1, #2 \rangle}
\newcommand{\der}[2]{\frac{\text{d}#2}{\text{d}#1}}
\newcommand{\pder}[2]{\frac{\partial#2}{\partial#1}}
\newcommand{\derTwo}[2]{\frac{\text{d}^2#2}{\text{d}#1^2}}
\newcommand{\ceil}[1]{\lceil #1 \rceil}
\newcommand{\Imag}[1]{\op{Im}\brac{ #1 }}
\newcommand{\Real}[1]{\op{Re}\brac{ #1 }}
\newcommand{\norm}[1]{\left|\left| #1 \right|\right| }
\newcommand{\norms}[1]{ \| #1 \|  }
\newcommand{\normProd}[1]{\left|\left| #1 \right|\right| _{\PROD{\cdot}{\cdot}} }
\newcommand{\normTwo}[1]{\left|\left| #1 \right|\right| _{2} }
\newcommand{\normTwos}[1]{ \| #1  \| _{2} }
\newcommand{\normZero}[1]{\left|\left| #1 \right|\right| _{0} }
\newcommand{\normTV}[1]{\left|\left| #1 \right|\right|  _{ \op{TV}  } }% _{\op{c} \ell_1} }
\newcommand{\normOne}[1]{\left|\left| #1 \right|\right| _{1} }
\newcommand{\normOnes}[1]{\| #1 \| _{1} }
\newcommand{\normOneTwo}[1]{\left|\left| #1 \right|\right| _{1,2} }
\newcommand{\normF}[1]{\left|\left| #1 \right|\right| _{\op{F}} }
\newcommand{\normLTwo}[1]{\left|\left| #1 \right|\right| _{\ml{L}_2} }
\newcommand{\normNuc}[1]{\left|\left| #1 \right|\right| _{\ast} }
\newcommand{\normOp}[1]{\left|\left| #1 \right|\right|  }
\newcommand{\normInf}[1]{\left|\left| #1 \right|\right| _{\infty}  }
\newcommand{\proj}[1]{\mathcal{P}_{#1} \, }
\newcommand{\diff}[1]{ \, \text{d}#1 }
\newcommand{\vc}[1]{\boldsymbol{\vec{#1}}}
\newcommand{\rc}[1]{\boldsymbol{#1}}
\newcommand{\vx}{\vec{x}}
\newcommand{\vy}{\vec{y}}
\newcommand{\vz}{\vec{z}}
\newcommand{\vu}{\vec{u}}
\newcommand{\vv}{\vec{v}}
\newcommand{\vb}{\vec{\beta}}
\newcommand{\va}{\vec{\alpha}}
\newcommand{\vaa}{\vec{a}}
\newcommand{\vbb}{\vec{b}}
\newcommand{\vg}{\vec{g}}
\newcommand{\vw}{\vec{w}}
\newcommand{\vh}{\vec{h}}
\newcommand{\vnu}{\vec{\nu}}
\newcommand{\rvnu}{\vc{\nu}}

\newtheorem{theorem}{Theorem}[section]
% \declaretheorem[style=plain,qed=$\square$]{theorem}
\newtheorem{corollary}[theorem]{Corollary}
\newtheorem{definition}[theorem]{Definition}
\newtheorem{lemma}[theorem]{Lemma}
\newtheorem{remark}[theorem]{Remark}
\newtheorem{algorithm}[theorem]{Algorithm}

% \theoremstyle{definition}
%\newtheorem{example}[proof]{Example}
%\declaretheorem[style=definition,qed=$\triangle$,sibling=definition]{example}
%\declaretheorem[style=definition,qed=$\bigcirc$,sibling=definition]{application}

%
%% Typographic tweaks and miscellaneous
%\newcommand{\sfrac}[2]{\mbox{\small$\displaystyle\frac{#1}{#2}$}}
%\newcommand{\suchthat}{\kern0.1em{:}\kern0.3em}
%\newcommand{\qqquad}{\kern3em}
%\newcommand{\cond}{\,|\,}
%\def\Matlab{\textsc{Matlab}}
%\newcommand{\displayskip}[1]{\abovedisplayskip #1\belowdisplayskip #1}
%\newcommand{\term}[1]{\emph{#1}}
%\renewcommand{\implies}{\;\Rightarrow\;}

% My macros

\def\Kset{\mathbb{K}}
\def\Nset{\mathbb{N}}
\def\Qset{\mathbb{Q}}
\def\Rset{\mathbb{R}}
\def\Sset{\mathbb{S}}
\def\Zset{\mathbb{Z}}
\def\squareforqed{\hbox{\rlap{$\sqcap$}$\sqcup$}}
\def\qed{\ifmmode\squareforqed\else{\unskip\nobreak\hfil
\penalty50\hskip1em\null\nobreak\hfil\squareforqed
\parfillskip=0pt\finalhyphendemerits=0\endgraf}\fi}

%\DeclareMathOperator*{\E}{\rm E}
%\DeclareMathOperator*{\argmax}{\rm argmax}
%\DeclareMathOperator*{\argmin}{\rm argmin}
%\DeclareMathOperator{\sgn}{sign}
\DeclareMathOperator{\supp}{supp}
\DeclareMathOperator{\last}{last}
%\DeclareMathOperator{\sign}{\sgn}
\DeclareMathOperator{\diag}{diag}
\providecommand{\abs}[1]{\lvert#1\rvert}
\providecommand{\norm}[1]{\lVert#1\rVert}
\def\vcdim{\textnormal{VCdim}}
\DeclareMathOperator*{\B}{\textbf{B}}

%\DeclarePairedDelimiter\ceil{\lceil}{\rceil}
%\DeclarePairedDelimiter\floor{\lfloor}{\rfloor}

\newcommand{\cX}{{\mathcal X}}
\newcommand{\cY}{{\mathcal Y}}
\newcommand{\cA}{{\mathcal A}}
\newcommand{\ignore}[1]{}
\newcommand{\bi}{\begin{itemize}}
\newcommand{\ei}{\end{itemize}}
\newcommand{\be}{\begin{enumerate}}
\newcommand{\ee}{\end{enumerate}}
\newcommand{\bd}{\begin{description}}
\newcommand{\ed}{\end{description}}
\newcommand{\h}{\widehat}
\newcommand{\e}{\epsilon}
\newcommand{\mat}[1]{{\mathbf #1}}
%\newcommand{\R}{\mat{R}}
\newcommand{\0}{\mat{0}}
\newcommand{\M}{\mat{M}}

\newcommand{\D}{\mat{D}}
\renewcommand{\r}{\mat{r}}
\newcommand{\x}{\mat{x}}
\renewcommand{\u}{\mat{u}}
\renewcommand{\v}{\mat{v}}
\newcommand{\w}{\mat{w}}
\renewcommand{\H}{\text{0}}
\newcommand{\T}{\text{1}}
%\newcommand{\set}[1]{\{#1\}}
\newcommand{\xxi}{{\boldsymbol \xi}}
\newcommand{\ssigma}{{\boldsymbol \sigma}}
\newcommand{\Alpha}{{\boldsymbol \alpha}}
\newcommand{\tts}{\tt \small}
\newcommand{\hint}{\emph{hint}}
\newcommand{\matr}[1]{\bm{#1}}     % ISO complying version
\newcommand{\vect}[1]{\bm{#1}} % vectors

%\newcommand{\Var}{\mathrm{Var}}
%\newcommand{\Cov}{\mathrm{Cov}}

% New commands
\newcommand{\SP}{\mathbf{S}_{+}^n}
\newcommand{\Proj}{\mathcal{P}_{\mathcal{S}}}
\DeclarePairedDelimiterX{\inp}[2]{\langle}{\rangle}{#1, #2}
\newtheorem{proof}{Proof}


\begin{document}

\begin{center}
{\large{\textbf{Homework 4}} } \vspace{0.2cm}\\
Due March 8 at 11 pm
\end{center}

\begin{enumerate}

\item (Condition number) Let $A\in \R^{n\times n}$ be invertible, and let $x_{\op{true}},y\in \R^n$
  satisfy $Ax_{\op{true}}=y$. We are interested in what happens if $y$ is perturbed additively by a vector $z\in \R^n$, i.e. if we solve 
\begin{align}
A w=y+z.
\end{align}
  \begin{enumerate}
  \item The operator norm of a matrix $M$ is equal to 
  \begin{align}
  \norm{M} := \arg \max_{\normTwo{v} = 1} \normTwo{Mv},
  \end{align}  
  which we know is equal to the maximum singular value. What is the operator norm of $A^{-1}$?
  \item Prove that $\|w-x_{\op{true}}\|\leq \|z\|/s_n$, where
    $s_j$ denotes the $j$th singular value of $A$.
  \item If $x_{\op{true}}\neq 0$ prove that
    $$\frac{\|w-x_{\op{true}}\|}{\|x_{\op{true}}\|} \leq
    \kappa(A)\frac{\|z\|}{\|y\|}.$$
    Here $\kappa(A):=s_1/s_n$ is called the \textit{condition
    number} of $A$.
  \end{enumerate} 
  
  \item (Simple linear regression) We consider a linear model with one feature ($p:=1$). The data are given by
\begin{align}
\ry_i : = x_i \beta + \rnd{z}_i, \quad 1 \leq i \leq n,
\end{align}
where $\beta \in \R$, $x_i \in \R$, and $\rz_1$, \ldots, $\rz_n$ are iid Gaussian random variables with zero mean and variance $\sigma^2$. A reasonable definition of the \emph{energy} in the feature is its sample mean square $\gamma^2 :=\frac{1}{n}\sum_{i=1}^{n}x_i^2$. We define the signal-to-noise ratio in the data as SNR$:= \gamma^2/\sigma^2$.
  \begin{enumerate}
  \item What is the distribution of the OLS estimate $\rnd{\beta}_{OLS}$ as a function of the SNR?
  \item If the SNR is fixed, how does the estimate behave as $n \rightarrow \infty$? If $n$ is fixed, how does the estimate behave as $\op{SNR} \rightarrow \infty$? Can this behavior change if the noise is iid, has zero mean and variance $\sigma^2$, but is not Gaussian? Prove that it doesn't or provide an example where it does.
  \item Can the behavior of the estimator as $n \rightarrow \infty$ change if the noise is not iid? Prove that it doesn't or provide a counterexample.
  \end{enumerate} 
   
\item (Best unbiased estimator) Consider the linear regression model
  $$\ry = X^T\beta + \rz$$
  where $\ry\in\R^n$, $X\in\R^{p \times n}$ has rank $p$,
  $\beta\in\R^p$, and $\rz\in\R^n$ has mean
  zero and covariance matrix $\Sigma_{z}=\sigma^2I$ for some
  $\sigma^2>0$.  Here only $\rz$ and $\ry$ are random.  We observe
  the values of $\ry$ and $X$ and must estimate $\beta$.
  Consider a linear estimator of the form $C\ry$
  where $C\in\R^{p\times n}$ (note that $X$ and $C$ are both
  deterministic, i.e., not random).
  \begin{enumerate}
  \item What is the mean $\mu=\E[C\ry]$?
  \item What is the covariance matrix of $C\ry$? That is, compute
    $$\E[(C\ry)(C\ry)^T]-\mu\mu^T.$$
  \item Write $C=(X^TX)^{-1}X^T+D$ for some $D\in\R^{p\times n}$.
    What must be true of $D$ so that $C\ry$ is an unbiased estimator
    of $\beta$ for all possible $\beta$?  That is, what must be true
    so that $\E[C\ry]=\beta$ for all $\beta$? [Hint: Use part (a).
    Your answer will be a property of $DX$.]
  \item Let $\Sigma_C$ denote the covariance matrix of $C\ry$ and let
    $\Sigma_{\text{OLS}}$ denote the covariance matrix of $(XX^T)^{-1}X \ry$.
    Show that if $C\ry$ is an unbiased estimator of $\beta$ then
    $$v^T\Sigma_Cv \geq v^T\Sigma_{\text{OLS}}v,$$
    for all $v\in\R^p$.  That is, least squares yields
    the estimator with smallest variance in any direction
    $v$. [Hint: Use part (b) to compute the covariance of
      $((XX^T)^{-1}X+D)\ry$.]
  \item Now suppose that the true regression model has extra features:
    $$\ry = X^T\beta + Z^T w + \rz,$$
    where $Z\in \R^{n\times k}$ and $w\in\R^k$.  Not knowing
    these features, you compute the least squares estimator
    $$\hat{\beta} = (XX^T)^{-1}X\ry.$$
    Under what conditions on $X,Z$ is $\hat{\beta}$ still unbiased for all
    possible $w$?
  \end{enumerate}   
   
\item (Distribution of $\beta$)  In this question, we will investigate how the coefficients of regression, $\beta$ is distributed. We will use the \href{https://archive.ics.uci.edu/ml/datasets/Combined+Cycle+Power+Plant}{combined cycle power plant data set} to regress for the net hourly electrical energy output as a function of the ambient temperature and exhaust vacuum. The support code loads the datasets and defines these subset of variables as $X$ and $y$ respectively. We will fit a regression to obtain $\beta_0, \beta$ which minimizes $y = \beta_0 + \beta^Tx$. 

To study the distribution of $\beta$, we split our dataset into $500$ bootstrap samples, each with $100$ data points. We fit linear regression individually on each of these $500$ bootstrap samples to obtain $\beta^1, \beta^2, \dots, \beta^{500}$.
\begin{enumerate}
\item Plot a histogram of the distribution of $\beta_1^k$ and $\beta_2^k$ where $k$ refers to the $k^{th}$ bootstrap sample and $\beta_i$ refers to the $i^{th}$ component of $\beta$.  The support code handles the actual plotting part, you only have to compute the $\beta^k$s. 
\item Make a scatter plot of $\beta_1^k$ vs $\beta_2^k$. Plot the principal directions of the actual data $X$ and the principal directions of $\beta^k$s. 
\item Do the principal directions of $X$ datapoints and $\beta^k$ datapoints align? Give a condition on the data generation process under which these principal directions will align. 
\end{enumerate}
\end{enumerate}
\end{document}
