\documentclass[12pt,twoside]{article}
\usepackage[dvipsnames]{xcolor}
\usepackage{tikz,graphicx,amsmath,amsfonts,amscd,amssymb,bm,cite,epsfig,epsf,url}
\usepackage[hang,flushmargin]{footmisc}
\usepackage[colorlinks=true,urlcolor=blue,citecolor=blue]{hyperref}
\usepackage{amsthm,multirow,wasysym,appendix}
\usepackage{array,subcaption} 
% \usepackage[small,bf]{caption}
\usepackage{bbm}
\usepackage{pgfplots}
\usetikzlibrary{spy}
\usepgfplotslibrary{external}
\usepgfplotslibrary{fillbetween}
\usetikzlibrary{arrows,automata}
\usepackage{thmtools}
\usepackage{blkarray} 
\usepackage{textcomp}
\usepackage[left=0.8in,right=1.0in,top=1.0in,bottom=1.0in]{geometry}

\usepackage{bm}
%% Probability operators and functions
%
% \def \P{\mathrm{P}}
%\input{Symbols.tex}
\def \P{\mathrm{P}}
\def \E{\mathrm{E}}
\def \Var{\mathrm{Var}}
\let\var\Var
\def \Cov {\mathrm{Cov}} \let\cov\Cov
\def \MSE {\mathrm{MSE}} \let\mse\MSE
\def \sgn {\mathrm{sgn}}
\def \R {\mathbb{R}}
\def \RR {\mathbb{R}}
\def \C {\mathbb{C}}
\def \CC {\mathbb{C}}
\def \N {\mathbb{N}}
\def \Z {\mathbb{Z}}
\def \cV {\mathcal{V}}
\def \cS {\mathcal{S}}
\DeclareMathOperator{\rank}{rank}
\DeclareMathOperator*{\argmin}{arg\,min}
\DeclareMathOperator*{\argmax}{arg\,max}
\newcommand{\red}[1]{\textcolor{red}{#1}}
\newcommand{\blue}[1]{\textcolor{blue}{#1}}
\newcommand{\green}[1]{\textcolor{ForestGreen}{ #1}}
\newcommand{\fuchsia}[1]{\textcolor{RoyalPurple}{ #1}}

%
%% Probability distributions
%
%\def \Bern    {\mathrm{Bern}}
%\def \Binom   {\mathrm{Binom}}
%\def \Exp     {\mathrm{Exp}}
%\def \Geom    {\mathrm{Geom}}
\def \Norm    {\mathcal{N}}
%\def \Poisson {\mathrm{Poisson}}
%\def \Unif    {\mathrm {U}}
%
\newcommand{\bdb}[1]{\textcolor{red}{#1}}

\newcommand{\ml}[1]{\mathcal{ #1 } }
\newcommand{\wh}[1]{\widehat{ #1 } }
\newcommand{\wt}[1]{\widetilde{ #1 } }
\newcommand{\conj}[1]{\overline{ #1 } }
\newcommand{\rnd}[1]{\tilde{ #1 } }
\newcommand{\rv}[1]{ \rnd{ #1}  }
\newcommand{\rvw}{\rv{w}  }
\newcommand{\rvx}{\rv{x}  }
\newcommand{\rvy}{\rv{y}  }
\newcommand{\rvz}{\rv{z}  }
\newcommand{\rvbeta}{\rv{\beta}  }
\newcommand{\rvu}{\rv{u}  }
\newcommand{\rx}{\rnd{ x}  }
\newcommand{\ry}{\rnd{ y}  }
\newcommand{\ra}{\rnd{ a}  }
\newcommand{\rb}{\rnd{ b}  }

\def \cnd {\, | \,}
\def \Id { I }

\newcommand{\op}[1]{\operatorname{#1}}
\newcommand{\setdef}[2]{ := \keys{ #1 \; | \; #2 } }
\newcommand{\set}[2]{ \keys{ #1 \; | \; #2 } }
\newcommand{\sign}[1]{\op{sign}\left( #1 \right) }
\newcommand{\trace}[1]{\op{tr}\left( #1 \right) }
\newcommand{\tr}[1]{\op{tr}\left( #1 \right) }
\newcommand{\inv}[1]{\left( #1 \right)^{-1} }
\newcommand{\abs}[1]{\left| #1 \right|}
\newcommand{\keys}[1]{\left\{ #1 \right\}}
\newcommand{\sqbr}[1]{\left[ #1 \right]}
\newcommand{\sbrac}[1]{ ( #1 ) }
\newcommand{\brac}[1]{\left( #1 \right) }
\newcommand{\bbrac}[1]{\big( #1 \big) }
\newcommand{\Bbrac}[1]{\Big( #1 \Big)}
\newcommand{\BBbrac}[1]{\BIG( #1 \Big)}
\newcommand{\MAT}[1]{\begin{bmatrix} #1 \end{bmatrix}}
\newcommand{\sMAT}[1]{\left(\begin{smallmatrix} #1 \end{smallmatrix}\right)}
\newcommand{\sMATn}[1]{\begin{smallmatrix} #1 \end{smallmatrix}}
\newcommand{\PROD}[2]{\left \langle #1, #2\right \rangle}
\newcommand{\PRODs}[2]{\langle #1, #2 \rangle}
\newcommand{\der}[2]{\frac{\text{d}#2}{\text{d}#1}}
\newcommand{\pder}[2]{\frac{\partial#2}{\partial#1}}
\newcommand{\derTwo}[2]{\frac{\text{d}^2#2}{\text{d}#1^2}}
\newcommand{\ceil}[1]{\lceil #1 \rceil}
\newcommand{\Imag}[1]{\op{Im}\brac{ #1 }}
\newcommand{\Real}[1]{\op{Re}\brac{ #1 }}
\newcommand{\norm}[1]{\left|\left| #1 \right|\right| }
\newcommand{\normProd}[1]{\left|\left| #1 \right|\right| _{\PROD{\cdot}{\cdot}} }
\newcommand{\normTwo}[1]{\left|\left| #1 \right|\right| _{2} }
\newcommand{\normTwos}[1]{ \| #1  \| _{2} }
\newcommand{\normZero}[1]{\left|\left| #1 \right|\right| _{0} }
\newcommand{\normTV}[1]{\left|\left| #1 \right|\right|  _{ \op{TV}  } }% _{\op{c} \ell_1} }
\newcommand{\normOne}[1]{\left|\left| #1 \right|\right| _{1} }
\newcommand{\normOneTwo}[1]{\left|\left| #1 \right|\right| _{1,2} }
\newcommand{\normF}[1]{\left|\left| #1 \right|\right| _{\op{F}} }
\newcommand{\normLTwo}[1]{\left|\left| #1 \right|\right| _{\ml{L}_2} }
\newcommand{\normNuc}[1]{\left|\left| #1 \right|\right| _{\ast} }
\newcommand{\normOp}[1]{\left|\left| #1 \right|\right|  }
\newcommand{\normInf}[1]{\left|\left| #1 \right|\right| _{\infty}  }
\newcommand{\proj}[1]{\mathcal{P}_{#1} \, }
\newcommand{\diff}[1]{ \, \text{d}#1 }
\newcommand{\vc}[1]{\boldsymbol{\vec{#1}}}
\newcommand{\rc}[1]{\boldsymbol{#1}}
\newcommand{\vx}{\vec{x}}
\newcommand{\vy}{\vec{y}}
\newcommand{\vz}{\vec{z}}
\newcommand{\vu}{\vec{u}}
\newcommand{\vv}{\vec{v}}
\newcommand{\ve}{\vec{e}}
\newcommand{\valpha}{\vec{\alpha}}
\newcommand{\vbeta}{\vec{\beta}}
\newcommand{\vgamma}{\vec{\gamma}}
\newcommand{\vb}{\vec{\beta}}
\newcommand{\va}{\vec{\alpha}}
\newcommand{\vaa}{\vec{a}}
\newcommand{\vbb}{\vec{b}}
\newcommand{\vg}{\vec{g}}
\newcommand{\vw}{\vec{w}}
\newcommand{\vh}{\vec{h}}
\newcommand{\vnu}{\vec{\nu}}
\newcommand{\rvnu}{\vc{\nu}}

\newtheorem{theorem}{Theorem}[section]
% \declaretheorem[style=plain,qed=$\square$]{theorem}
\newtheorem{corollary}[theorem]{Corollary}
\newtheorem{definition}[theorem]{Definition}
\newtheorem{lemma}[theorem]{Lemma}
\newtheorem{remark}[theorem]{Remark}
\newtheorem{algorithm}[theorem]{Algorithm}


%
%% Typographic tweaks and miscellaneous
%\newcommand{\sfrac}[2]{\mbox{\small$\displaystyle\frac{#1}{#2}$}}
%\newcommand{\suchthat}{\kern0.1em{:}\kern0.3em}
%\newcommand{\qqquad}{\kern3em}
%\newcommand{\cond}{\,|\,}
%\def\Matlab{\textsc{Matlab}}
%\newcommand{\displayskip}[1]{\abovedisplayskip #1\belowdisplayskip #1}
%\newcommand{\term}[1]{\emph{#1}}
%\renewcommand{\implies}{\;\Rightarrow\;}



\begin{document}

\begin{center}
{\large{\textbf{Homework 7}} } \vspace{0.2cm}\\
Due April 12 at 11 pm
\end{center}

\begin{enumerate}

\item (Fourier coefficients and smoothness) Let $x:\R\to\C$ be
  periodic with period $1$
  and let $\hat{x}[k]$ denote the $k$th Fourier coefficient of $x$,
  for $k\in\Z$ (computed on any interval of length $1$).
  \begin{enumerate}
  \item Suppose $x$ is continuously differentiable. Prove that for
    $k\neq 0$ we have
    $$|\hat{x}[k]| \leq \frac{C_1}{|k|}$$
    for some $C_1\geq0$ that depends on $x$ (but not on $k$). [Hint:
      Integration by parts.  Also note that
      $$\left|\int_0^1 f(t)\,dt\right| \leq \int_0^1 |f(t)|\,dt<\infty$$
    if $f$ is continuous on $[0,1]$.]
  \item Suppose $x$ is twice continuously differentiable.
    Prove that for $k\neq 0$ we have
    $$|\hat{x}[k]| \leq \frac{C_2}{|k|^2}$$
    for some $C_2\geq0$ that depends on $x$ (but not on $k$).
  \end{enumerate}
  
\item (Sampling a sum of sinusoids) We are interested in a signal $x$ belonging to the unit interval $[0,1]$ of the form
\begin{align}
x(t) := a_1 \exp (i 2 \pi k_1 t ) + a_2 \exp (i 2 \pi k_2 t ),
\end{align}
where the amplitudes $a_1$ and $a_2$ are complex numbers, and the
frequencies $k_1$ and $k_2$ are known integers.
We sample the signal at $N$ equispaced locations $0$, $1/N$, $2/N$,
\ldots, $(N-1)/N$, for some positive integer $N$.  
\begin{enumerate}
\item What value of $N$ is required by the Sampling Theorem to
  guarantee that we can reconstruct $x$ from the samples? 
\item Write a system of equations in matrix form mapping the
  amplitudes $a_1$ and $a_2$ to the samples $x_{N}$. 
  \item Under what condition on $N$, $k_1$ and $k_2$ can we recover the
  amplitudes from the samples by solving the system of equations? 
  Can $N$ be smaller than the value dictated by the Sampling Theorem?
  If yes, give an example. If not, explain why.  
  \item What is the limitation of this approach, which could make it unrealistic?
\end{enumerate}
   
\item (Sampling theorem for bandpass signals) Bandpass signals are signals that have nonzero Fourier coefficients
only in a fixed band of the frequency domain. We are interested in
sampling a bandpass signal $x$ belonging to the unit interval $[0,1]$
that has nonzero Fourier-series coefficients between $k_1$ and $k_2$,
inclusive, where $k_1$ and $k_2$ are known positive integers such that $k_2 > k_1$. 
\begin{enumerate}
\item We sample the signal at $N$ equispaced locations $0$, $1/N$, $2/N$, \ldots, $(N-1)/N$. What value of $N$ is required by the Sampling Theorem to guarantee that we can reconstruct $x$ from the samples?
\item Assume that $k_2:=k_1 + 2\tilde{k}_c$, where $\tilde{k}_c $ is a positive integer. For any $N \geq 2\tilde{k}_c + 1$ it is possible to recover the signal from the samples. Explain why (you don't need to derive any explicit expressions).
\item Assume that $k_2:=k_1 + 2\tilde{k}_c$, $N\geq 2\tilde{k}_c+1$,
  and $mN = k_1+\tilde{k}_c$ for some integer $m$. Explain precisely how to recover $x$ from the samples in this case.
\end{enumerate} 
 
 \item  (Frequency analysis of musical notes) In this exercise you will
  use the code and data in the \texttt{musicdata} folder.  Make sure you have
  the python packages sklearn, pandas, sounddevice, and soundfile
  installed.  The skeleton code for you to work with is given in
  \texttt{analysis.py} which uses tools given in
  \texttt{music\_tools.py}.  The data used here comes from the NSynth
  dataset.

  \begin{enumerate}
  \item Plot the audio signals for the first signal in the training
    set, and the first vocal signal in the training set (i.e., the
    first signal whose \texttt{instrument\_family\_str} field is
    'vocal' in the dataframe).  In the titles of your two plots, include the
    \texttt{instrument\_family\_str} and the frequency (in Hz).
    We recommend you also use \texttt{play\_signal}
    to hear what the signals sound like.
  \item For each signal in the test set, compute the (strictly positive)
    frequency with the largest amplitude (in absolute value), and
    convert it to a pitch number (using the tools in
    \texttt{music\_tools}).  This will be our predicted pitch.
    \begin{enumerate}
    \item Report what overall fraction of the signals in
      the test set you accurately predict using this method (i.e.,
      your overall accuracy).
    \item For the
      first two signals you misclassify (in the order they occur in the
      test set), give plots of their absolute
      DFT coefficients (use \texttt{np.fft.fft} and make one plot per
      signal). In the title of your plots, include the
      \texttt{instrument\_family\_str}, the true frequency, and the
      predicted frequency (in Hz).  Make
      sure to plot the coefficients on an axis centered at $0$ by using
      fftfreq with the correct arguments.
    \item What is the instrument family for which the method got the
      highest fraction of incorrect predictions (i.e., number incorrect
      divided by number of examples from that family)?
    \end{enumerate}
  \item Use the \texttt{LogisticRegression} class in sklearn to fit a
    pitch classifier on the training set using the absolute DFT
    coefficients as the features.  Use the default parameters
    but set \texttt{multi\_class} to 'multinomial' and
    \texttt{solver} to 'lbfgs'.  Note: We will use the negative
    frequencies as well for convenience, even though they have the same magnitudes as
    the positive (the $L_2$ regularization will take care of it for us).
    \begin{enumerate}
    \item Report your score on the test set as computed by the model.
    \item Give 3 plots of the model coefficients for pitches 60, 65, and 72.
      Make sure to plot the coefficients on an axis centered at $0$ by using
      fftfreq with the correct arguments (because the coefficients
      correspond to frequencies).
    \end{enumerate}
  \end{enumerate}

 \end{enumerate}
\end{document}
