\documentclass[12pt,twoside]{article}
\usepackage[dvipsnames]{xcolor}
\usepackage{tikz,graphicx,amsmath,amsfonts,amscd,amssymb,bm,cite,epsfig,epsf,url}
\usepackage[hang,flushmargin]{footmisc}
\usepackage[colorlinks=true,urlcolor=blue,citecolor=blue]{hyperref}
\usepackage{amsthm,multirow,wasysym,appendix}
\usepackage{array,subcaption} 
% \usepackage[small,bf]{caption}
\usepackage{bbm}
\usepackage{pgfplots}
\usetikzlibrary{spy}
\usepgfplotslibrary{external}
\usepgfplotslibrary{fillbetween}
\usetikzlibrary{arrows,automata}
\usepackage{thmtools}
\usepackage{blkarray} 
\usepackage{textcomp}
\usepackage[left=0.8in,right=1.0in,top=1.0in,bottom=1.0in]{geometry}

\usepackage{times}
\usepackage{amsfonts}
\usepackage{amsmath}
%\usepackage[psamsfonts]{amssymb}
\usepackage{latexsym}
\usepackage{color}
\usepackage{graphics}
\usepackage{enumerate}
\usepackage{amstext}
\usepackage{blkarray}
\usepackage{url}
\usepackage{epsfig}
\usepackage{bm}
\usepackage{hyperref}
\hypersetup{
    colorlinks=true,
    linkcolor=blue,
    filecolor=magenta,      
    urlcolor=blue,
}
\usepackage{mathtools}


%% Probability operators and functions
%
% \def \P{\mathrm{P}}
\def \P{\mathrm{P}}
\def \E{\mathrm{E}}
\def \Var{\mathrm{Var}}
\let\var\Var
\def \Cov {\mathrm{Cov}} \let\cov\Cov
\def \MSE {\mathrm{MSE}} \let\mse\MSE
\def \sgn {\mathrm{sgn}}
\def \R {\mathbb{R}}
\def \C {\mathbb{C}}
\def \N {\mathbb{N}}
\def \Z {\mathbb{Z}}
\def \cV {\mathcal{V}}
\def \cS {\mathcal{S}}
\DeclareMathOperator*{\argmin}{arg\,min}
\DeclareMathOperator*{\argmax}{arg\,max}
\newcommand{\red}[1]{\textcolor{red}{#1}}
\newcommand{\blue}[1]{\textcolor{blue}{#1}}
\newcommand{\green}[1]{\textcolor{ForestGreen}{ #1}}
\newcommand{\fuchsia}[1]{\textcolor{RoyalPurple}{ #1}}

%
%% Probability distributions
%
%\def \Bern    {\mathrm{Bern}}
%\def \Binom   {\mathrm{Binom}}
%\def \Exp     {\mathrm{Exp}}
%\def \Geom    {\mathrm{Geom}}
%\def \Norm    {\mathcal{N}}
%\def \Poisson {\mathrm{Poisson}}
%\def \Unif    {\mathrm {U}}
%
\newcommand{\bdb}[1]{\textcolor{red}{#1}}

\newcommand{\ml}[1]{\mathcal{ #1 } }
\newcommand{\wh}[1]{\widehat{ #1 } }
\newcommand{\wt}[1]{\widetilde{ #1 } }
\newcommand{\conj}[1]{\overline{ #1 } }
\newcommand{\rnd}[1]{\tilde{ #1 } }
\newcommand{\rv}[1]{ \rnd{ #1}  }
\newcommand{\rx}{\rnd{ x}  }
\newcommand{\ry}{\rnd{ y}  }
\newcommand{\ra}{\rnd{ a}  }
\newcommand{\rb}{\rnd{ b}  }
\newcommand{\rpc}{\widetilde{ pc}  }

\def \cnd {\, | \,}
\def \Id { I }
\def \J {\mathbf{1}\mathbf{1}^T}

\newcommand{\op}[1]{\operatorname{#1}}
\newcommand{\setdef}[2]{ := \keys{ #1 \; | \; #2 } }
\newcommand{\set}[2]{ \keys{ #1 \; | \; #2 } }
\newcommand{\sign}[1]{\op{sign}\left( #1 \right) }
\newcommand{\trace}[1]{\op{tr}\left( #1 \right) }
\newcommand{\tr}[1]{\op{tr}\left( #1 \right) }
\newcommand{\inv}[1]{\left( #1 \right)^{-1} }
\newcommand{\abs}[1]{\left| #1 \right|}
\newcommand{\sabs}[1]{| #1 |}
\newcommand{\keys}[1]{\left\{ #1 \right\}}
\newcommand{\sqbr}[1]{\left[ #1 \right]}
\newcommand{\sbrac}[1]{ ( #1 ) }
\newcommand{\brac}[1]{\left( #1 \right) }
\newcommand{\bbrac}[1]{\big( #1 \big) }
\newcommand{\Bbrac}[1]{\Big( #1 \Big)}
\newcommand{\BBbrac}[1]{\BIG( #1 \Big)}
\newcommand{\MAT}[1]{\begin{bmatrix} #1 \end{bmatrix}}
\newcommand{\sMAT}[1]{\left(\begin{smallmatrix} #1 \end{smallmatrix}\right)}
\newcommand{\sMATn}[1]{\begin{smallmatrix} #1 \end{smallmatrix}}
\newcommand{\PROD}[2]{\left \langle #1, #2\right \rangle}
\newcommand{\PRODs}[2]{\langle #1, #2 \rangle}
\newcommand{\der}[2]{\frac{\text{d}#2}{\text{d}#1}}
\newcommand{\pder}[2]{\frac{\partial#2}{\partial#1}}
\newcommand{\derTwo}[2]{\frac{\text{d}^2#2}{\text{d}#1^2}}
\newcommand{\ceil}[1]{\lceil #1 \rceil}
\newcommand{\Imag}[1]{\op{Im}\brac{ #1 }}
\newcommand{\Real}[1]{\op{Re}\brac{ #1 }}
\newcommand{\norm}[1]{\left|\left| #1 \right|\right| }
\newcommand{\norms}[1]{ \| #1 \|  }
\newcommand{\normProd}[1]{\left|\left| #1 \right|\right| _{\PROD{\cdot}{\cdot}} }
\newcommand{\normTwo}[1]{\left|\left| #1 \right|\right| _{2} }
\newcommand{\normTwos}[1]{ \| #1  \| _{2} }
\newcommand{\normZero}[1]{\left|\left| #1 \right|\right| _{0} }
\newcommand{\normTV}[1]{\left|\left| #1 \right|\right|  _{ \op{TV}  } }% _{\op{c} \ell_1} }
\newcommand{\normOne}[1]{\left|\left| #1 \right|\right| _{1} }
\newcommand{\normOnes}[1]{\| #1 \| _{1} }
\newcommand{\normOneTwo}[1]{\left|\left| #1 \right|\right| _{1,2} }
\newcommand{\normF}[1]{\left|\left| #1 \right|\right| _{\op{F}} }
\newcommand{\normLTwo}[1]{\left|\left| #1 \right|\right| _{\ml{L}_2} }
\newcommand{\normNuc}[1]{\left|\left| #1 \right|\right| _{\ast} }
\newcommand{\normOp}[1]{\left|\left| #1 \right|\right|  }
\newcommand{\normInf}[1]{\left|\left| #1 \right|\right| _{\infty}  }
\newcommand{\proj}[1]{\mathcal{P}_{#1} \, }
\newcommand{\diff}[1]{ \, \text{d}#1 }
\newcommand{\vc}[1]{\boldsymbol{\vec{#1}}}
\newcommand{\rc}[1]{\boldsymbol{#1}}
\newcommand{\vx}{\vec{x}}
\newcommand{\vy}{\vec{y}}
\newcommand{\vz}{\vec{z}}
\newcommand{\vu}{\vec{u}}
\newcommand{\vv}{\vec{v}}
\newcommand{\vb}{\vec{\beta}}
\newcommand{\va}{\vec{\alpha}}
\newcommand{\vaa}{\vec{a}}
\newcommand{\vbb}{\vec{b}}
\newcommand{\vg}{\vec{g}}
\newcommand{\vw}{\vec{w}}
\newcommand{\vh}{\vec{h}}
\newcommand{\vnu}{\vec{\nu}}
\newcommand{\rvnu}{\vc{\nu}}

\newtheorem{theorem}{Theorem}[section]
% \declaretheorem[style=plain,qed=$\square$]{theorem}
\newtheorem{corollary}[theorem]{Corollary}
\newtheorem{definition}[theorem]{Definition}
\newtheorem{lemma}[theorem]{Lemma}
\newtheorem{remark}[theorem]{Remark}
\newtheorem{algorithm}[theorem]{Algorithm}

% \theoremstyle{definition}
%\newtheorem{example}[proof]{Example}
%\declaretheorem[style=definition,qed=$\triangle$,sibling=definition]{example}
%\declaretheorem[style=definition,qed=$\bigcirc$,sibling=definition]{application}

%
%% Typographic tweaks and miscellaneous
%\newcommand{\sfrac}[2]{\mbox{\small$\displaystyle\frac{#1}{#2}$}}
%\newcommand{\suchthat}{\kern0.1em{:}\kern0.3em}
%\newcommand{\qqquad}{\kern3em}
%\newcommand{\cond}{\,|\,}
%\def\Matlab{\textsc{Matlab}}
%\newcommand{\displayskip}[1]{\abovedisplayskip #1\belowdisplayskip #1}
%\newcommand{\term}[1]{\emph{#1}}
%\renewcommand{\implies}{\;\Rightarrow\;}

% My macros

\def\Kset{\mathbb{K}}
\def\Nset{\mathbb{N}}
\def\Qset{\mathbb{Q}}
\def\Rset{\mathbb{R}}
\def\Sset{\mathbb{S}}
\def\Zset{\mathbb{Z}}
\def\squareforqed{\hbox{\rlap{$\sqcap$}$\sqcup$}}
\def\qed{\ifmmode\squareforqed\else{\unskip\nobreak\hfil
\penalty50\hskip1em\null\nobreak\hfil\squareforqed
\parfillskip=0pt\finalhyphendemerits=0\endgraf}\fi}

%\DeclareMathOperator*{\E}{\rm E}
%\DeclareMathOperator*{\argmax}{\rm argmax}
%\DeclareMathOperator*{\argmin}{\rm argmin}
%\DeclareMathOperator{\sgn}{sign}
\DeclareMathOperator{\supp}{supp}
\DeclareMathOperator{\last}{last}
%\DeclareMathOperator{\sign}{\sgn}
\DeclareMathOperator{\diag}{diag}
\providecommand{\abs}[1]{\lvert#1\rvert}
\providecommand{\norm}[1]{\lVert#1\rVert}
\def\vcdim{\textnormal{VCdim}}
\DeclareMathOperator*{\B}{\textbf{B}}

%\DeclarePairedDelimiter\ceil{\lceil}{\rceil}
%\DeclarePairedDelimiter\floor{\lfloor}{\rfloor}

\newcommand{\cX}{{\mathcal X}}
\newcommand{\cY}{{\mathcal Y}}
\newcommand{\cA}{{\mathcal A}}
\newcommand{\ignore}[1]{}
\newcommand{\bi}{\begin{itemize}}
\newcommand{\ei}{\end{itemize}}
\newcommand{\be}{\begin{enumerate}}
\newcommand{\ee}{\end{enumerate}}
\newcommand{\bd}{\begin{description}}
\newcommand{\ed}{\end{description}}
\newcommand{\h}{\widehat}
\newcommand{\e}{\epsilon}
\newcommand{\mat}[1]{{\mathbf #1}}
%\newcommand{\R}{\mat{R}}
\newcommand{\0}{\mat{0}}
\newcommand{\M}{\mat{M}}

\newcommand{\D}{\mat{D}}
\renewcommand{\r}{\mat{r}}
\newcommand{\x}{\mat{x}}
\renewcommand{\u}{\mat{u}}
\renewcommand{\v}{\mat{v}}
\newcommand{\w}{\mat{w}}
\renewcommand{\H}{\text{0}}
\newcommand{\T}{\text{1}}
%\newcommand{\set}[1]{\{#1\}}
\newcommand{\xxi}{{\boldsymbol \xi}}
\newcommand{\ssigma}{{\boldsymbol \sigma}}
\newcommand{\Alpha}{{\boldsymbol \alpha}}
\newcommand{\tts}{\tt \small}
\newcommand{\hint}{\emph{hint}}
\newcommand{\matr}[1]{\bm{#1}}     % ISO complying version
\newcommand{\vect}[1]{\bm{#1}} % vectors

%\newcommand{\Var}{\mathrm{Var}}
%\newcommand{\Cov}{\mathrm{Cov}}

% New commands
\newcommand{\SP}{\mathbf{S}_{+}^n}
\newcommand{\Proj}{\mathcal{P}_{\mathcal{S}}}
\DeclarePairedDelimiterX{\inp}[2]{\langle}{\rangle}{#1, #2}
\newtheorem{proof}{Proof}


\begin{document}

\noindent DS-GA.1013 Mathematical Tools for Data Science \\
Homework 1 \\
Yves Greatti - yg390\\


\begin{enumerate}
\item (Rotation) For a symmetric matrix $A$, can there be a nonzero vector $x$ such that $Ax$ is nonzero and orthogonal to $x$? Either prove that this is impossible, or explain under what condition on the eigenvalues of $A$ such a vector exists.
Let $x \in V, x \neq 0$, an inner product space , by the spectral theorem there exists an orthonormal basis of  $V$, consisting of eigenvectors of $A$, let $u_1$, \ldots, $u_n$ be the eigenbasis of $A$, and $\lambda_1, \ldots, \lambda_n$ the eigenvalues for each of these eigenvectors.
$x \in \text{span}\{u_1, \ldots, u_n \} \Rightarrow  x=\sum_{i=1,n} \alpha_i u_i, \alpha_i \neq 0$.  $x^T (Ax) = (\sum_{i=1,n} \alpha_i u_i) (\sum_{j=1,n} \alpha_j A u_j) =  (\sum_{i=1,n} \alpha_i u_i) (\sum_{j=1,n} \alpha_j \lambda_j u_j) = \sum_{i=1,n} \alpha_i^2 \lambda_i$ since $u_i^T u_j = 0$ for $i \neq j$ and $u_i^T u_i =1$. $Ax$ is orthogonal to $x$: $x^T (Ax) = 0 \Rightarrow  \sum_{i=1,n} \alpha_i^2 \lambda_i = 0$.

\item (Matrix decomposition) The trace can be used to define an inner product between matrices:
\begin{align}
\PROD{A}{B} := \trace{A^TB}, \quad A,B \in \R^{m \times n},
\end{align}
where the corresponding norm is the Frobenius norm $\normF{A}:=\PROD{A}{A}$.
\begin{enumerate}
\item Express the inner product in terms of vectorized matrices and use the result to prove that this is a valid inner product.
$(A B)_{ij} = (\sum_k A_{ik} B_{kj})_{ij}$, and $(A^TB) _{ij} = (\sum_k A_{ki} B_{kj})_{ij}$.
$\trace{A} = \sum_i A_{ii}  \Rightarrow \trace{A^TB} = \sum_i \sum_k A_{ki} B_{ki} = \sum_i \sum_j A_{ij} B_{ij} = \text{vec}(A)^T \text{vect}(B) = \PROD{\text{vec}(A)} {\text{vec}(B)}$.
The trace is then the inner product between vectors in $\R^{mn}$ thus is a valid inner product.

\item Prove that for any $A,B \in \R^{m \times n}$, $\trace{A^TB}=\trace{BA^T}$.
$\trace{BA^T} =  \sum_i \sum_k B_{ik} A_{ik} =  \sum_i \sum_j A_{ij} B_{ij} =  \trace{A^TB}$.

\item Let $u_1$, \ldots, $u_n$ be the eigenvectors of a symmetric matrix $A$. Compute the inner product between the rank-1 matrices $u_iu_i^T$ and $u_ju_j^T$ for $i \neq j$, and also the norm of $u_iu_i^T$ for $i=1,\ldots,n$. 
For $i \neq j$, $\PROD{u_iu_i^T}{u_ju_j^T} = \trace{u_iu_i^Tu_ju_j^T} = 0$, if $u_i, u_j$ are two eigenvectors for different eigenvalues.
if $i=j$ then  $\PROD{u_iu_i^T}{u_iu_i^T} = \trace{u_iu_i^Tu_iu_i^T} = \trace{(u_i^Tu_i)^2} = (u_i^Tu_i)^2 \Rightarrow |\normF{u_i u_i^T} =  u_i^Tu_i$ and $\normF{u_iu_i^T}=1$ if the eigenvectors are also orthonormal.
\item What is the projection of $A$ onto $u_iu_i^T$?
The projection of $A$ onto $u_iu_i^T$ is $\PROD{A}{u_iu_i^T}$. If $A$ is a symmetric matrix, by the spectral theorem, $A=U D U^T= \sum_i \lambda_i u_i u_i^T$ where $\lambda_i, i=1, \ldots,n$ are the eigenvalues of $A$. If $u_1$, \ldots, $u_n$ form an eigenbasis then $\PROD{A}{u_iu_i^T} = \lambda_i$. $u_iu_i^T$ is the matrix for the orthogonal projection
onto $\text{ span }(u_i)$.

\item Provide a geometric interpretation of the matrix $A':=A-\lambda_1 u_1u_1^T$, which we defined in the proof of the spectral theorem, based on your previous answers.
From the previous question the orthogonal projection of A in $u_iu_i^T$ is $\lambda_i u_iu_i^T$ so $A' = \sum_i \lambda_i u_iu_i^T, i \neq 1$ has row or column subspaces contained in  $(u_1)^\bot$.

\end{enumerate}

\item (Quadratic forms) Let $A\in \R^{n \times n}$ be a symmetric matrix, and let $f(x):=x^TAx$ be the corresponding quadratic form. We consider the 1D function $g_{v}(t)=f(tv)$ obtained by restricting the quadratic form to lie in the direction of a vector $v$ with unit $\ell_2$ norm.
\begin{enumerate}
\item Is $g_{v}(t)$ a polynomial? If so, what kind?
	$g_{v}(t) = f(tv) = (tv)^T A (tv) = t^2 v^T A v = v^T A v \; t^2$, $g_{v}(t)$ is a second-order polynomial in $t$. 
\item What is the curvature (i.e. the second derivative) of $g_{v}(t)=f(tv)$ at an arbitrary point $t$?
$g'_{v}(t)= 2 v^T A v \; t$ and the curvature is $g''_{v}(t)= 2 v^T A v$

\item What are the directions of maximum and minimum curvature of the quadratic form? What are the corresponding curvatures equal to?
By the spectral theorem, $A = U \textbf{diag}(\lambda) U^T$ where $\lambda_1 \ge \lambda_2 \ge \ldots \lambda_n$ are the eigenvalues and $u_1, \dots, u_n$ the eigenvectors.
$\lambda_1 = \max_{\|v\|_2 =1} v^T A v$ and $u_1 = \arg \max_{\|v\|_2 =1} v^T A v$, and the minimum is given by $\lambda_n = \max_{\|v\|_2 =1} v^T A v, u_n = \arg \max_{\|v\|_2 =1} v^T A v$. Thus the maximum curvature is given by the largest eigenvalue $\lambda_1$ and  is in the direction of the corresponding eigenvector $u_1$. 
	The smallest curvature is  given  by  the  smallest  eigenvalue $\lambda_n$ and is in the direction of the corresponding eigenvector $u_n$.

\end{enumerate}

\item (Projected gradient ascent) Projected gradient descent is a method designed to find the maximum of a differentiable function $f:\R^n \rightarrow \R$ in a constraint set $\ml{S}$. Let $\ml{P}_{\ml{S}}$ denote the projection onto $\ml{S}$, i.e.
\begin{align}
\ml{P}_{\ml{S}}(x) := \arg \min_{y \in \ml{S}} \normTwo{x-y}^2.
\end{align} 
The $k$th update of projected gradient ascent equals
\begin{align}
x^{[k]} :=\ml{P}_{\ml{S}}( x^{[k-1]} + \alpha \nabla f (x^{[k-1]}) ), \qquad k=1,2,\ldots,
\end{align}
where $\alpha$ is a positive constant and $x^{[0]}$ is an arbitrary initial point.
\begin{enumerate}
\item Use the same arguments we used to prove Lemmas 5.1 and 5.2 in the notes on PCA to derive the projection of a vector $x$ onto the unit sphere in $n$ dimensions.
\item Derive an algorithm based on projected gradient ascent to find the maximum eigenvalue of a symmetric matrix $A\in \R^{n \times n}$.
\item Let us express the iterations in the basis of eigenvectors of $A$: $x^{[k]} := \sum_{i=1}^{n}\beta_i^{[k]} u_i$. Compute the ratio between the coefficient corresponding to the largest eigenvalue and the rest $\frac{\beta_1^{[k]}}{\beta_i^{[k]}}$ as a function of $k$, $\alpha$, and $\beta_1^{[0]}$, \ldots, $\beta_n^{[0]}$. Under what conditions on $\alpha$ and the initial point does the algorithm converge to the eigenvector $u_1$ corresponding to the largest eigenvalue? What happens if $\alpha$ is extremely large (i.e. when $\alpha \rightarrow \infty$)?
\item Implement the algorithm derived in part (b). Support code is provided in {\tt main.py} within {\tt Q4.zip}. Observe what happens for different sizes of $\alpha$. Report the plots generated by the script.
\end{enumerate}

\end{enumerate}

\end{document}
